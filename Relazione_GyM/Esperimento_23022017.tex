\chapter{Calcolo della densità e della temperatura elettronica per un plasma di Argon attraverso due differenti tecniche: sonda di Langmuir e modello spettroscopico}
\chaptermark{Plasma di Argon}


L'obiettivo di questo esperimento è quello di verificare l'effettiva validità del modello spettroscopico per calcolare la densità e la temperatura elettronica di un plasma di Argon.\\
\`E stato fatto un confronto tra i dati acquisiti tramite sonda di Langmuir e tramite spettroscopia ottica per verificarne la conformità. Si è quindi visto che 
l'applicazione del modello spettroscopico è un buon metodo sostitutivo di diagnostica nei casi in cui la sonda di Langmuir non è idonea: per esempio, vi sono casi in cui è necessario 
avere una particolare configurazione del campo magnetico che non sempre garantisce la perpendicolarità tra le linee del suddetto e l'area di acquisizione della sonda. In quest'ultimo
caso gli errori sui dati acquisiti sarebbero grandi e risulta quindi necessario avere un sistema di diagnostica differente quale è la spettroscopia ottica.

\section{Strumenti utilizzati e il loro funzionamento: sonda di Langmuir e spettroscopio ottico}
\label{sec:strumenti}
\subsection{Sonda di Langmuir}
\subsection{Spettroscopio ottico}

\section{Condizioni sperimentali e analisi dati}
Le condizioni sperimentali utilizzate sono le seguenti:
\begin{itemize}
 \item corrente delle bobine di GyM fissa a $600\si{\ampere}$;
 \item due differenti flussi di Argon, $0.5sccm$, con pressione dei neutri a plasma spento di $P_n=2.40\cdot10^{-5}\si{\milli\bar}$, e $3.0sccm$, con pressione dei neutri a plasma spento di $P_n=1.34\cdot10^{-4}\si{\milli\bar}$;
 \item tre diverse potenze per le microonde, $20\%$, $40\%$, $60\%$.
\end{itemize}

\paragraph{Osservazioni} ~\\
I primi dati raccolti in laboratorio sono stati quelli a $\text{Flusso}_\text{Ar}=0.5\text{sccm}$ e con potenza delle microonde al $40\%$. In questi dati si può notare la mancanza della pressione dei neutri durante l'acquisizione degli spettri. Successivamente si è deciso di registrare anche $P_{neutri}$ per tenere monitorata l'intera condione di acquisizione dati.\\
Le variazioni di potenza delle microonde sono state effettuate riportando di volta in volta il gas alla condizione di "non-plasma" e aspettando lo stabilizzarsi della pressione in camera al valore registrato precedente alla prima accensione del plasma.\\
Gli accorgimenti di qui sopra sono stati fatti per mantenere le condizioni del sistema il più possibile invariate, così da poter confrontare i dati raccolti a diversi flussi e potenze in modo consistente.\\
L'accensione del plasma, nel caso di plasma di Argon, è visibile anche attraverso una salita della pressione in camera che raggiunge un \textit{plateau} dopo qualche secondo dall'accensione delle microonde.

\subsection{Analisi dati: profilo di densità a diverse pressioni e potenze}

GRAFICI E COMMENTI
\subsection{Analisi dati: profili di temperatura a diverse pressioni e potenze}

GRAFICI E COMMENTI

\subsection{Descrizione del programma utilizzato per l'analisi dati}
In Allegato(VA FATTO RIFERIMENTO) viene riportato il programma scritto per l'analisi dei dati.\\
DESCRIVIAMO A GRANDI LINEE COSA FA



\section{Conclusioni finali}
BLA BLA

\section{File dati}
\subsection{File utilizzati per Flusso di Argon a $0.5sccm$}

\paragraph*{Potenza microonde $\text{20\%}$} ~\\
Caratteristiche tensione-corrente della sonda di Langmuir:
\begin{center}
\begin{tabular}{p{3cm}p{3cm}}
\toprule
File name	&$P_{n}$ [$\si{\milli\bar}$]\\
\midrule
$2302lang04$	&$4.16\cdot10^{-5}$\\
$2302lang05$	&$3.84\cdot10^{-5}$\\
\bottomrule
\end{tabular}
\end{center}

Spettri raccolti:
\begin{center}
\begin{tabular}{p{3cm}p{4cm}p{2cm}p{3cm}}
\toprule
File name	&$\lambda_\text{range}\text{/}\lambda_\text{centre}$ [nm] &AT [s]\footnote{$\text{AT}=\text{acquisition time}$} &$P_{n}$ [$\si{\milli\bar}$]\\
\midrule
$2302speca10$	&$300-600$	&$0.5$		&$3.91\cdot10^{-5}$\\
$2302speca11$	&$600-800$	&$0.01$		&$3.88\cdot10^{-5}$\\
$2302speca12$	&$480$		&$0.5$		&$3.88\cdot10^{-5}$\\
$2302speca13$	&$480$		&$0.5$		&$3.88\cdot10^{-5}$\\
$2302speca14$	&$480$		&$0.5$		&$3.84\cdot10^{-5}$\\
$2302speca15$	&$750$		&$0.01$		&$3.84\cdot10^{-5}$\\
$2302speca16$	&$750$		&$0.01$		&$3.84\cdot10^{-5}$\\
$2302speca17$	&$750$		&$0.01$		&$3.84\cdot10^{-5}$\\
\bottomrule
\end{tabular}
\end{center}

\paragraph*{Potenza microonde $\text{40\%}$} ~\\
Caratteristiche tensione-corrente della sonda di Langmuir:
\begin{center}
  \begin{tabular}{p{3cm}p{3cm}}
  \toprule
File name	&$P_{n}$ [$\si{\milli\bar}$]\\
  \midrule
$2302lang01$	&$6.59\cdot10^{-5}$\\
$2302lang02$	&$5.81\cdot10^{-5}$\\
$2302lang03$	&$5.66\cdot10^{-5}$\\
  \bottomrule
  \end{tabular}
\end{center}

Spettri raccolti:
\begin{center}
\begin{tabular}{p{3cm}p{3.5cm}p{3.5cm}}
\toprule
File name	&$\lambda_\text{range}\text{/}\lambda_\text{centre}$ [nm] &AT [s]\footnote{$\text{AT}=\text{acquisition time}$}\\
\midrule
$2302speca1$	&$300-600$	&$0.5$\\
$2302speca2$	&$600-800$	&$0.01$\\
$2302speca3$	&$480$		&$0.5$\\
$2302speca4$	&$750$		&$0.01$\\
$2302speca5$	&$750$		&$0.01$\\
$2302speca6$	&$750$		&$0.01$\\
$2302speca7$	&$480$		&$0.5$\\
$2302speca8$	&$480$		&$0.5$\\
\bottomrule
\end{tabular}
\end{center}

\paragraph*{Potenza microonde $\text{60\%}$} ~\\
Caratteristiche tensione-corrente della sonda di Langmuir:
\begin{center}
\begin{tabular}{p{3cm}p{3cm}}
\toprule
File name	&$P_n$ [$\si{\milli\bar}$]\\
\midrule
$2302lang06$	&$3.15\cdot10^{-5}$\\
$2302lang07$	&$3.13\cdot10^{-5}$\\
\bottomrule
\end{tabular}
\end{center}

Spettri raccolti:
\begin{center}
\begin{tabular}{p{3cm}p{4cm}p{2cm}p{3cm}}
\toprule
File name	&$\lambda_\text{range}\text{/}\lambda_\text{centre}$ [nm] 	&AT [s]\footnote{$\text{AT}=\text{acquisition time}$} &$P_n$ [$\si{\milli\bar}$]\\
\midrule
$2302speca18$	&$300-600$	&$0.5$		&$3.09\cdot10^{-5}$\\
$2302speca19$	&$600-800$	&$0.01$		&$3.09\cdot10^{-5}$\\
$2302speca20$	&$480$		&$0.5$		&$3.09\cdot10^{-5}$\\
$2302speca21$	&$480$		&$0.5$		&$3.09\cdot10^{-5}$\\
$2302speca22$	&$480$		&$0.5$		&$3.09\cdot10^{-5}$\\
$2302speca23$	&$750$		&$0.01$		&$3.12\cdot10^{-5}$\\
$2302speca24$	&$750$		&$0.01$		&$3.12\cdot10^{-5}$\\
$2302speca25$	&$750$		&$0.01$		&$3.12\cdot10^{-5}$\\
\bottomrule
\end{tabular}
\end{center}



\subsection{File utilizzati per Flusso di Argon a $3.0sccm$}

\paragraph*{Potenza microonde $\text{20\%}$} ~\\
Caratteristiche tensione-corrente della sonda di Langmuir:
\begin{center}
\begin{tabular}{p{3cm}p{3cm}}
\toprule
File name	&$P_{n}$ [$\si{\milli\bar}$]\\
\midrule
$2302lang08$	&$1.46\cdot10^{-4}$\\
$2302lang09$	&$1.47\cdot10^{-4}$\\
\bottomrule
\end{tabular}
\end{center}

Spettri raccolti:
\begin{center}
\begin{tabular}{p{3cm}p{4cm}p{2cm}p{3cm}}
\toprule
File name	&$\lambda_\text{range}\text{/}\lambda_\text{centre}$ [nm] &AT [s]\footnote{$\text{AT}=\text{acquisition time}$} &$P_{n}$ [$\si{\milli\bar}$]\\
\midrule
$2302speca26$	&$300-600$	&$1.0$		&$1.47\cdot10^{-4}$\\
$2302speca27$	&$600-800$	&$0.01$		&$1.42\cdot10^{-4}$\\
$2302speca28$	&$480$		&$1.0$		&$1.47\cdot10^{-4}$\\
$2302speca29$	&$480$		&$1.0$		&$1.47\cdot10^{-4}$\\
$2302speca30$	&$480$		&$1.0$		&$1.47\cdot10^{-4}$\\
$2302speca31$	&$750$		&$0.01$		&$1.47\cdot10^{-4}$\\
$2302speca32$	&$750$		&$0.01$		&$1.47\cdot10^{-4}$\\
$2302speca33$	&$750$		&$0.01$		&$1.47\cdot10^{-4}$\\
\bottomrule
\end{tabular}
\end{center}


\paragraph*{Potenza microonde $\text{40\%}$} ~\\
Caratteristiche tensione-corrente della sonda di Langmuir:
\begin{center}
\begin{tabular}{p{3cm}p{3cm}}
\toprule
File name	&$P_{n}$ [$\si{\milli\bar}$]\\
\midrule
$2302lang10$	&$1.54\cdot10^{-4}$\\
$2302lang11$	&$1.55\cdot10^{-4}$\\
\bottomrule
\end{tabular}
\end{center}

Spettri raccolti:
\begin{center}
\begin{tabular}{p{3cm}p{4cm}p{2cm}p{3cm}}
\toprule
File name	&$\lambda_\text{range}\text{/}\lambda_\text{centre}$ [nm] &AT [s]\footnote{$\text{AT}=\text{acquisition time}$} &$P_{n}$ [$\si{\milli\bar}$]\\					&\\
\midrule
$2302speca34$	&$300-600$	&$0.5$		&$1.55\cdot10^{-4}$\\
$2302speca35$	&$600-800$	&$0.01$		&$1.55\cdot10^{-4}$\\
$2302speca36$	&$480$		&$0.5$		&$1.55\cdot10^{-4}$\\
$2302speca37$	&$480$		&$0.5$		&$1.55\cdot10^{-4}$\\
$2302speca38$	&$480$		&$0.5$		&$1.55\cdot10^{-4}$\\
$2302speca39$	&$750$		&$0.01$		&$1.55\cdot10^{-4}$\\
$2302speca40$	&$750$		&$0.01$		&$1.55\cdot10^{-4}$\\
$2302speca41$	&$750$		&$0.01$		&$1.55\cdot10^{-4}$\\
\bottomrule
\end{tabular}
\end{center}

\paragraph*{Potenza microonde $\text{60\%}$} ~\\
Caratteristiche tensione-corrente della sonda di Langmuir:
\begin{center}
\begin{tabular}{p{3cm}p{3cm}}
\toprule
File name	&$P_{n}$ [$\si{\milli\bar}$]\\
\midrule
$2302lang12$	&$1.56\cdot10^{-4}$\\
$2302lang13$	&$1.56\cdot10^{-4}$\\
\bottomrule
\end{tabular}
\end{center}

Spettri raccolti:
\begin{center}
\begin{tabular}{p{3cm}p{4cm}p{2cm}p{3cm}}
\toprule
File name	&$\lambda_\text{range}\text{/}\lambda_\text{centre}$ [nm] &AT [s]\footnote{$\text{AT}=\text{acquisition time}$} &$P_{n}$ [$\si{\milli\bar}$]\\					&\\
\midrule
$2302speca42$	&$300-600$	&$0.5$		&$1.56\cdot10^{-5}$\\
$2302speca43$	&$600-800$	&$0.01$		&$1.56\cdot10^{-5}$\\
$2302speca44$	&$480$		&$0.5$		&$1.56\cdot10^{-5}$\\
$2302speca45$	&$480$		&$0.5$		&$1.56\cdot10^{-5}$\\
$2302speca46$	&$480$		&$0.5$		&$1.56\cdot10^{-5}$\\
$2302speca47$	&$750$		&$0.01$		&$1.56\cdot10^{-5}$\\
$2302speca48$	&$750$		&$0.01$		&$1.56\cdot10^{-5}$\\
$2302speca49$	&$750$		&$0.01$		&$1.56\cdot10^{-5}$\\
\bottomrule
\end{tabular}
\end{center}



