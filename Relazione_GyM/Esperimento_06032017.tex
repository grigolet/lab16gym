\chapter{Calcolo della densità e della temperatura elettronica per un plasma di Argon, Deuterio e Azoto attraverso due differenti tecniche: sonda di Langmuir e modello spettroscopico}
\chaptermark{Plasma di Argon, Deuterio e Azoto}

L'obiettivo di questo esperimento è il medesimo degli esperimenti precedenti ma questa volta lo studio viene effettuato su un plasma di Argon, Deuterio e Azoto.

\section{Strumenti utilizzati e il loro funzionamento: sonda di Langmuir e spettroscopio ottico}
Per descrizione dettagliata degli strumenti vedi capitolo precedente, \textit{Sezione}~\ref{sec:strumenti}.

\section{Condizioni sperimentali e analisi dati}
Le condizioni sperimentali utilizzate sono le seguenti:
\begin{itemize}
 \item corrente delle bobine di GyM fissa a $600\si{\ampere}$;
 
 VANNO FATTI I CALCOLI DELLE PERCENTUALI E POI CORRETTE tutte LE PRESSIONI CON I FATTORI CORETTIVI rispettivi
 
 \item due differenti condizioni di miscela\\
 $\text{Ar}=  $,  $\text{Ar}+\text{N}_2=  $, $\text{Ar}+\text{N}_2+\text{De}=  $\\
 $\text{Ar}=  $,  $\text{Ar}+\text{N}_2=  $, $\text{Ar}+\text{N}_2+\text{De}=  $\\
 le pressioni riportate sono quelle a plasma spento;
 
 \item diverse potenze per le microonde, $20\%$, $30\%$, $40\%$, $50\%$, $60\%$ e $90\%$.
\end{itemize}


\paragraph{Osservazioni}
La sonda di Langmuir si trova a $2\si{\centi\meter}$ dal centro macchina.
In questo esperimento le variazioni di potenza delle microonde non sono state effettuate riportando di volta in volta il gas alla condizione di "non-plasma" ma aspettando semplicemente lo stabilizzarsi 
della pressione una volta impostata la potenza desiderata.\\
Lo studio spettrocopico verrà effettutato sul Argon e Deuterio anche perchè l'Azotosi dissolve difficilmente e si possono vedere al massimo delle piccolo linee atomiche.

\subsection{Analisi dati: profilo di densità a diverse pressioni e potenze}

GRAFICI E COMMENTI
\subsection{Analisi dati: profili di temperatura a diverse pressioni e potenze}

GRAFICI E COMMENTI

\subsection{Descrizione del programma utilizzato per l'analisi dati}
In Allegato(VA FATTO RIFERIMENTO) viene riportato il programma scritto per l'analisi dei dati.\\
DESCRIVIAMO A GRANDI LINEE COSA FA


\section{Conclusioni finali}
BLA BLA

\section{File dati}
MANCANO TUTTE LE PRESSIONI

\subsection{File utilizzati per pressione totale di $   $}

\paragraph*{Potenza microonde $\text{20\%}$} ~\\
Caratteristiche tensione-corrente della sonda di Langmuir:
\begin{center}
\begin{tabular}{p{3cm}p{3cm}}
\toprule
File name	&$P_{n}$ [$\si{\milli\bar}$]\\
\midrule
$0603lang01$	&$   $\\
$0603lang02$	&$   $\\
$0603lang03$	&$   $\\
\bottomrule
\end{tabular}
\end{center}

Spettri raccolti:
\begin{center}
\begin{tabular}{p{3cm}p{4cm}p{2cm}p{3cm}}
\toprule
File name	&$\lambda_\text{range}\text{/}\lambda_\text{centre}$ [nm] &AT [s]\footnote{$\text{AT}=\text{acquisition time}$} &$P_{n}$ [$\si{\milli\bar}$]\\
\midrule
$0603specnad11$	&$300-600$	&$0.2$		&$  $\\
$0603specde5$	&$300-600$	&$0.2$		&$  $\\
$0603specnad12$	&$600-800$	&$0.05$		&$  $\\
$0603specnad13$	&$600-800$	&$0.05$		&$  $\\

\bottomrule
\end{tabular}
\end{center}

\paragraph*{Potenza microonde $\text{30\%}$} ~\\
Caratteristiche tensione-corrente della sonda di Langmuir:
\begin{center}
  \begin{tabular}{p{3cm}p{3cm}}
  \toprule
File name	&$P_{n}$ [$\si{\milli\bar}$]\\
  \midrule
$0603lang04$	&$  $\\
$0603lang05$	&$  $\\

  \bottomrule
  \end{tabular}
\end{center}

Spettri raccolti:
\begin{center}
\begin{tabular}{p{3cm}p{4cm}p{2cm}p{3cm}}
\toprule
File name	&$\lambda_\text{range}\text{/}\lambda_\text{centre}$ [nm] &AT [s]\footnote{$\text{AT}=\text{acquisition time}$} &$P_{n}$ [$\si{\milli\bar}$]\\
\midrule
$0603specnad14$	&$300-600$	&$0.2$		&$  $\\
$0603specnad15$	&$300-600$	&$0.2$		&$  $\\
$0603specnad16$	&$600-800$	&$0.05$		&$  $\\
$0603specnad17$	&$600-800$	&$0.05$		&$  $\\


\bottomrule
\end{tabular}
\end{center}

\paragraph*{Potenza microonde $\text{40\%}$} ~\\
Caratteristiche tensione-corrente della sonda di Langmuir:
\begin{center}
\begin{tabular}{p{3cm}p{3cm}}
\toprule
File name	&$P_n$ [$\si{\milli\bar}$]\\
\midrule
$0603lang06$	&$  $\\
$0603lang07$	&$  $\\
\bottomrule
\end{tabular}
\end{center}

Spettri raccolti:
\begin{center}
\begin{tabular}{p{3cm}p{4cm}p{2cm}p{3cm}}
\toprule
File name	&$\lambda_\text{range}\text{/}\lambda_\text{centre}$ [nm] 	&AT [s]\footnote{$\text{AT}=\text{acquisition time}$} &$P_n$ [$\si{\milli\bar}$]\\
\midrule
$0603specnad18$	&$300-600$	&$0.2$		&$  $\\
$0603specnad19$	&$300-600$	&$0.2$		&$  $\\
$0603specnad20$	&$600-800$	&$0.05$		&$  $\\
$0603specnad21$	&$600-800$	&$0.05$		&$  $\\

\bottomrule
\end{tabular}
\end{center}

\paragraph*{Potenza microonde $\text{50\%}$} ~\\
Caratteristiche tensione-corrente della sonda di Langmuir:
\begin{center}
\begin{tabular}{p{3cm}p{3cm}}
\toprule
File name	&$P_n$ [$\si{\milli\bar}$]\\
\midrule
$0603lang08$	&$  $\\
$0603lang09$	&$  $\\
\bottomrule
\end{tabular}
\end{center}

Spettri raccolti:
\begin{center}
\begin{tabular}{p{3cm}p{4cm}p{2cm}p{3cm}}
\toprule
File name	&$\lambda_\text{range}\text{/}\lambda_\text{centre}$ [nm] 	&AT [s]\footnote{$\text{AT}=\text{acquisition time}$} &$P_n$ [$\si{\milli\bar}$]\\
\midrule
$0603specnad22$	&$300-600$	&$0.2$		&$  $\\
$0603specnad23$	&$300-600$	&$0.2$		&$  $\\
$0603specnad24$	&$600-800$	&$0.05$		&$  $\\
$0603specnad25$	&$600-800$	&$0.05$		&$  $\\

\bottomrule
\end{tabular}
\end{center}

\paragraph*{Potenza microonde $\text{60\%}$} ~\\
Caratteristiche tensione-corrente della sonda di Langmuir:
\begin{center}
\begin{tabular}{p{3cm}p{3cm}}
\toprule
File name	&$P_n$ [$\si{\milli\bar}$]\\
\midrule
$0603lang10$	&$  $\\
$0603lang11$	&$  $\\
\bottomrule
\end{tabular}
\end{center}

Spettri raccolti:
\begin{center}
\begin{tabular}{p{3cm}p{4cm}p{2cm}p{3cm}}
\toprule
File name	&$\lambda_\text{range}\text{/}\lambda_\text{centre}$ [nm] 	&AT [s]\footnote{$\text{AT}=\text{acquisition time}$} &$P_n$ [$\si{\milli\bar}$]\\
\midrule
$0603specnad26$	&$300-600$	&$0.2$		&$  $\\
$0603specnad27$	&$300-600$	&$0.2$		&$  $\\
$0603specnad28$	&$600-800$	&$0.05$		&$  $\\
$0603specnad29$	&$600-800$	&$0.05$		&$  $\\

\bottomrule
\end{tabular}
\end{center}

\paragraph*{Potenza microonde $\text{90\%}$} ~\\
Caratteristiche tensione-corrente della sonda di Langmuir:
\begin{center}
\begin{tabular}{p{3cm}p{3cm}}
\toprule
File name	&$P_n$ [$\si{\milli\bar}$]\\
\midrule
$0603lang12$	&$  $\\
$0603lang13$	&$  $\\
\bottomrule
\end{tabular}
\end{center}

Spettri raccolti:
\begin{center}
\begin{tabular}{p{3cm}p{4cm}p{2cm}p{3cm}}
\toprule
File name	&$\lambda_\text{range}\text{/}\lambda_\text{centre}$ [nm] 	&AT [s]\footnote{$\text{AT}=\text{acquisition time}$} &$P_n$ [$\si{\milli\bar}$]\\
\midrule
$0603specnad30$	&$300-600$	&$0.2$		&$  $\\
$0603specnad31$	&$600-800$	&$0.05$		&$  $\\


\bottomrule
\end{tabular}
\end{center}

\subsection{File utilizzati per pressione totale di $  $}

\paragraph*{Potenza microonde $\text{20\%}$}~\\
Caratteristiche tensione-corrente della sonda di Langmuir:
\begin{center}
\begin{tabular}{p{3cm}p{3cm}}
\toprule
File name	&$P_{n}$ [$\si{\milli\bar}$]\\
\midrule
$0603lang14$	&$  $\\

\bottomrule
\end{tabular}
\end{center}

Spettri raccolti:
\begin{center}
\begin{tabular}{p{3cm}p{4cm}p{2cm}p{3cm}}
\toprule
File name	&$\lambda_\text{range}\text{/}\lambda_\text{centre}$ [nm] &AT [s]\footnote{$\text{AT}=\text{acquisition time}$} &$P_{n}$ [$\si{\milli\bar}$]\\
\midrule
$0603specnad32$	&$600-800$	&$0.05$		&$  $\\
$0603specnad33$	&$300-600$	&$0.2$		&$  $\\


\bottomrule
\end{tabular}
\end{center}

\paragraph*{Potenza microonde $\text{30\%}$} ~\\
Caratteristiche tensione-corrente della sonda di Langmuir:
\begin{center}
  \begin{tabular}{p{3cm}p{3cm}}
  \toprule
File name	&$P_{n}$ [$\si{\milli\bar}$]\\
  \midrule
$0603lang16$	&$  $\\
$0603lang17$	&$  $\\

  \bottomrule
  \end{tabular}
\end{center}

Spettri raccolti:
\begin{center}
\begin{tabular}{p{3cm}p{4cm}p{2cm}p{3cm}}
\toprule
File name	&$\lambda_\text{range}\text{/}\lambda_\text{centre}$ [nm] &AT [s]\footnote{$\text{AT}=\text{acquisition time}$} &$P_{n}$ [$\si{\milli\bar}$]\\
\midrule
$0603specnad34$	&$300-600$	&$0.2$		&$  $\\
$0603specnad35$	&$300-600$	&$0.2$		&$  $\\
$0603specnad36$	&$600-800$	&$0.05$		&$  $\\
$0603specnad37$	&$600-800$	&$0.05$		&$  $\\


\bottomrule
\end{tabular}
\end{center}

\paragraph*{Potenza microonde $\text{40\%}$} ~\\
Caratteristiche tensione-corrente della sonda di Langmuir:
\begin{center}
\begin{tabular}{p{3cm}p{3cm}}
\toprule
File name	&$P_n$ [$\si{\milli\bar}$]\\
\midrule
$0603lang18$	&$  $\\
$0603lang19$	&$  $\\
\bottomrule
\end{tabular}
\end{center}

Spettri raccolti:
\begin{center}
\begin{tabular}{p{3cm}p{4cm}p{2cm}p{3cm}}
\toprule
File name	&$\lambda_\text{range}\text{/}\lambda_\text{centre}$ [nm] 	&AT [s]\footnote{$\text{AT}=\text{acquisition time}$} &$P_n$ [$\si{\milli\bar}$]\\
\midrule
$0603specnad38$	&$300-600$	&$0.2$		&$  $\\
$0603specnad39$	&$300-600$	&$0.2$		&$  $\\
$0603specnad40$	&$600-800$	&$0.05$		&$  $\\
$0603specnad41$	&$600-800$	&$0.05$		&$  $\\

\bottomrule
\end{tabular}
\end{center}

\paragraph*{Potenza microonde $\text{50\%}$} ~\\
Caratteristiche tensione-corrente della sonda di Langmuir:
\begin{center}
\begin{tabular}{p{3cm}p{3cm}}
\toprule
File name	&$P_n$ [$\si{\milli\bar}$]\\
\midrule
$0603lang20$	&$  $\\
$0603lang21$	&$  $\\
\bottomrule
\end{tabular}
\end{center}

Spettri raccolti:
\begin{center}
\begin{tabular}{p{3cm}p{4cm}p{2cm}p{3cm}}
\toprule
File name	&$\lambda_\text{range}\text{/}\lambda_\text{centre}$ [nm] 	&AT [s]\footnote{$\text{AT}=\text{acquisition time}$} &$P_n$ [$\si{\milli\bar}$]\\
\midrule
$0603specnad42$	&$300-600$	&$0.2$		&$  $\\
$0603specnad43$	&$300-600$	&$0.2$		&$  $\\
$0603specnad44$	&$600-800$	&$0.05$		&$  $\\
$0603specnad45$	&$600-800$	&$0.05$		&$  $\\

\end{tabular}
\end{center}

\paragraph*{Potenza microonde $\text{60\%}$} ~\\
Caratteristiche tensione-corrente della sonda di Langmuir:
\begin{center}
\begin{tabular}{p{3cm}p{3cm}}
\toprule
File name	&$P_n$ [$\si{\milli\bar}$]\\
\midrule
$0603lang22$	&$  $\\
$0603lang23$	&$  $\\
\bottomrule
\end{tabular}
\end{center}

Spettri raccolti:
\begin{center}
\begin{tabular}{p{3cm}p{4cm}p{2cm}p{3cm}}
\toprule
File name	&$\lambda_\text{range}\text{/}\lambda_\text{centre}$ [nm] 	&AT [s]\footnote{$\text{AT}=\text{acquisition time}$} &$P_n$ [$\si{\milli\bar}$]\\
\midrule
$0603specnad46$	&$300-600$	&$0.2$		&$  $\\
$0603specnad47$	&$300-600$	&$0.2$		&$  $\\
$0603specnad48$	&$600-800$	&$0.05$		&$  $\\
$0603specnad49$	&$600-800$	&$0.05$		&$  $\\

\bottomrule
\end{tabular}
\end{center}

\paragraph*{Potenza microonde $\text{90\%}$} ~\\
Caratteristiche tensione-corrente della sonda di Langmuir:
\begin{center}
\begin{tabular}{p{3cm}p{3cm}}
\toprule
File name	&$P_n$ [$\si{\milli\bar}$]\\
\midrule
$0603lang24$	&$  $\\
$0603lang25$	&$  $\\
\bottomrule
\end{tabular}
\end{center}

Spettri raccolti:
\begin{center}
\begin{tabular}{p{3cm}p{4cm}p{2cm}p{3cm}}
\toprule
File name	&$\lambda_\text{range}\text{/}\lambda_\text{centre}$ [nm] 	&AT [s]\footnote{$\text{AT}=\text{acquisition time}$} &$P_n$ [$\si{\milli\bar}$]\\
\midrule
$0603specnad50$	&$300-600$	&$0.2$		&$  $\\
$0603specnad51$	&$300-600$	&$0.2$		&$  $\\
$0603specnad52$	&$600-800$	&$0.05$		&$  $\\
$0603specnad53$	&$600-800$	&$0.05$		&$  $\\

\bottomrule
\end{tabular}
\end{center}