
\documentclass[a4paper,11pt]{book}
\usepackage{enumerate}
\usepackage[utf8]{inputenc}
\usepackage{csquotes}
\usepackage{lipsum} 
\usepackage[italian]{babel}
\usepackage[T1]{fontenc}
\usepackage{siunitx}
\usepackage{graphicx}
\usepackage{textcomp}
\usepackage{verbatim}
%\usepackage{pslatex}
%\usepackage{pstricks}
%cross reference footnote
\usepackage{cleveref}[2012/02/15]
%\crefformat{footnote}{#2\footnotemark[#1]#3}
\usepackage{graphicx}
\usepackage[colorlinks=true, linkcolor=black, urlcolor=black]{hyperref}
\usepackage[usenames,dvipsnames,svgnames,table]{xcolor}
\usepackage{xspace}
\usepackage{amsmath}
\usepackage{subfig}
\usepackage{float}
\usepackage{caption}
%\usepackage[Glenn]{fncychap}
\captionsetup[table]{position=top}
\usepackage{booktabs}
\usepackage{colortbl,array}
\usepackage[version=3]{mhchem}
\usepackage{listings} %per includere il codice
\usepackage{titling}
\usepackage{geometry}
\usepackage{emptypage}

\geometry{a4paper, top=3cm, bottom=4cm, left=3cm, right=3cm, heightrounded, bindingoffset=5mm}

\lstdefinestyle{custompy}{
  belowcaptionskip=1\baselineskip,
  breaklines=true,
  frame=L,
  xleftmargin=\parindent,
  language=Python,
  showstringspaces=false,
  basicstyle=\footnotesize\ttfamily,
  keywordstyle=\bfseries\color{green!40!black},
  commentstyle=\itshape\color{BlueGreen},
  identifierstyle=\color{blue},
  stringstyle=\color{BrickRed},
}
\lstset{
  numbers=left,
  stepnumber=1,    
  firstnumber=1,
  numberfirstline=true
}
\usepackage{indentfirst}
\lstset{language=Python,style=custompy}
\newcommand{\comando}[1]{\texttt{#1}}
\renewcommand\captionfont{\footnotesize} 
\renewcommand\captionlabelfont{\bfseries}
\usepackage{parskip}
\setlength{\parindent}{1em}
\setlength{\parskip}{1em}
\usepackage{soul}
\usepackage[opacity=.8, open=true]{pdfcomment}
\setulcolor{green}
\newcommand{\itememph}[1]{\textbf{#1}}
%non andare a capo su figura, sezione, appendice
\usepackage{hyphenat}
\hyphenation{figura sezione appendice}
%footnote tra parentesi
\makeatletter
\def\@makefnmark{\hbox{\@textsuperscript{\normalfont(\@thefnmark)}}}
\makeatother

\usepackage{setspace}
\onehalfspacing

\usepackage{newlfont}
\vspace{\stretch{2}}
\usepackage{titlesec}
\titleformat{\chapter}[display]
{\normalfont\bfseries\filcenter}
{\LARGE\thechapter}
{1ex}
{\titlerule[2pt]
\vspace{2ex}%
\LARGE}
[\vspace{1ex}%
{\titlerule[2pt]}]

\begin{document}

\frontmatter
\begin{center}
{\large \center \bf{Università degli Studi di Milano-Bicocca}}
 \center{\includegraphics[width=4cm]{logo} }
{\large \center\textsc{Facoltà di Scienze Matematiche, Fisiche e Naturali}
{ \Large \center {Corso di Laurea Magistrale in Fisica}}}
\vspace{2cm}
\centering
\hrule
\center {\bf {\textsc{Relazioni degli esperimenti svolti presso il CNR}}}
\vspace{\baselineskip}
\hrule
\vfill

\centering{
\normalsize{
\emph{Relazioni di:}\\
Magatti Demetrio, Muscente Paola e Rigoletti Gianluca\\\vspace{.1cm}}}

\vfill

\vspace{\baselineskip}

	{Anno Accademico 2016/2017}
\end{center} 

\tableofcontents 
\mainmatter
\include{Esperimento_23022017}
\chapter{Calcolo della densità e della temperatura elettronica per un plasma di Deuterio attraverso due differenti tecniche: sonda di Langmuir e modello spettroscopico}
\chaptermark{Plasma di Deuterio}


L'obiettivo di questo esperimento è il medesimo dell'esperimento precedente ma questa volta lo studio viene effettuato su un plasma di Deuterio.

\section{Strumenti utilizzati e il loro funzionamento: sonda di Langmuir e spettroscopio ottico}
Per descrizione dettagliata degli strumenti vedi capitolo precedente, \textit{Sezione}~\ref{sec:strumenti}.

\section{Condizioni sperimentali e analisi dati}
Le condizioni sperimentali utilizzate sono le seguenti:
\begin{itemize}
 \item corrente delle bobine di GyM fissa a $600\si{\ampere}$;
 \item due differenti pressioni di Deuterio, $4.08\cdot10^{-4}mbar$ (circa $45sccm$ di flusso) e $9.31\cdot10^{-5}mbar$ (circa $10sccm$ di flusso);
 \item diverse potenze per le microonde, $20\%$, $30\%$, $40\%$, $50\%$ e $60\%$.
\end{itemize}


\paragraph{Osservazioni} ~\\
A differenza dell'esperimento precedente, si è deciso di mantenere costanti le due pressioni di gas a plasma spento, con valori scelti di $4.08\cdot10^{-4}mbar$ e $9.31\cdot10^{-5}mbar$,
piuttosto che mantenerne fisso il flusso. 
Per mantenere le pressioni tali si sono operate a volte delle leggere variazioni di flusso di Deuterio.\\
Le variazioni di potenza delle microonde sono state effettuate riportando di volta in volta il gas alla condizione di "non-plasma" e aspettando lo stabilizzarsi della pressione ai due valori
indicati predentemente per uno e l'altro caso.\\
Gli accorgimenti di qui sopra sono stati fatti per mantenere le condizioni del sistema il più possibile invariate, così da poter confrontare i dati raccolti a diversi flussi e potenze in modo consistente.\\
L'accensione del plasma, nel caso di plasma di Deuterio, è visibile anche attraverso una discesa seguita da una salita della pressione in camera che raggiunge un \textit{plateau} dopo qualche secondo dall'accensione delle microonde.
A bassa pressione e bassa potenza delle microonde, $9.31\cdot10^{-5}$ con $20\%$ di potenza, si è visto che l'accensione del plasma avviene senza un abbassamento precedente alla salita della pressione.

\subsection{Analisi dati: profilo di densità a diverse pressioni e potenze}

GRAFICI E COMMENTI
\subsection{Analisi dati: profili di temperatura a diverse pressioni e potenze}

GRAFICI E COMMENTI

\subsection{Descrizione del programma utilizzato per l'analisi dati}
In Allegato(VA FATTO RIFERIMENTO) viene riportato il programma scritto per l'analisi dei dati.\\
DESCRIVIAMO A GRANDI LINEE COSA FA


\section{Conclusioni finali}
BLA BLA

\section{File dati}
\subsection{File utilizzati per pressione di Deuterio a $4.08\cdot10^{-4}mbar$}

\paragraph*{Potenza microonde $\text{20\%}$} ~\\
Caratteristiche tensione-corrente della sonda di Langmuir:
\begin{center}
\begin{tabular}{p{3cm}p{3cm}}
\toprule
File name	&$P_{n}$ [$\si{\milli\bar}$]\\
\midrule
$2702lang01$	&$4.15\cdot10^{-4}$\\
\bottomrule
\end{tabular}
\end{center}

Spettri raccolti:
\begin{center}
\begin{tabular}{p{3cm}p{4cm}p{2cm}p{3cm}}
\toprule
File name	&$\lambda_\text{range}\text{/}\lambda_\text{centre}$ [nm] &AT [s]\footnote{$\text{AT}=\text{acquisition time}$} &$P_{n}$ [$\si{\milli\bar}$]\\
\midrule
$2702specde1$	&$400-500$	&$0.2$		&$4.20\cdot10^{-4}$\\
$2702specde2$	&$400-500$	&$0.2$		&$4.20\cdot10^{-4}$\\
$2702specde3$	&$400-500$	&$0.2$		&$4.20\cdot10^{-4}$\\
$2702specde4$	&$656$		&$0.05$		&$4.20\cdot10^{-4}$\\
$2702specde5$	&$656$		&$0.05$		&$4.20\cdot10^{-4}$\\
$2702specde6$	&$656$		&$0.05$		&$4.20\cdot10^{-4}$\\

\bottomrule
\end{tabular}
\end{center}

\paragraph*{Potenza microonde $\text{30\%}$} ~\\
Caratteristiche tensione-corrente della sonda di Langmuir:
\begin{center}
  \begin{tabular}{p{3cm}p{3cm}}
  \toprule
File name	&$P_{n}$ [$\si{\milli\bar}$]\\
  \midrule
$2702lang16$	&$4.51\cdot10^{-4}$\\
$2702lang17$	&$4.51\cdot10^{-4}$\\

  \bottomrule
  \end{tabular}
\end{center}

Spettri raccolti:
\begin{center}
\begin{tabular}{p{3cm}p{4cm}p{2cm}p{3cm}}
\toprule
File name	&$\lambda_\text{range}\text{/}\lambda_\text{centre}$ [nm] &AT [s]\footnote{$\text{AT}=\text{acquisition time}$} &$P_{n}$ [$\si{\milli\bar}$]\\
\midrule
$2702specde49$	&$400-500$	&$0.2$		&$4.49\cdot10^{-4}$\\
$2702specde50$	&$400-500$	&$0.2$		&$4.51\cdot10^{-4}$\\
$2702specde51$	&$400-500$	&$0.2$		&$4.51\cdot10^{-4}$\\
$2702specde52$	&$656$		&$0.05$		&$4.51\cdot10^{-4}$\\
$2702specde53$	&$656$		&$0.05$		&$4.51\cdot10^{-4}$\\
$2702specde54$	&$656$		&$0.05$		&$4.51\cdot10^{-4}$\\

\bottomrule
\end{tabular}
\end{center}

\paragraph*{Potenza microonde $\text{40\%}$} ~\\
Caratteristiche tensione-corrente della sonda di Langmuir:
\begin{center}
\begin{tabular}{p{3cm}p{3cm}}
\toprule
File name	&$P_n$ [$\si{\milli\bar}$]\\
\midrule
$2702lang02$	&$4.27\cdot10^{-4}$\\
$2702lang03$	&$4.32\cdot10^{-4}$\\
\bottomrule
\end{tabular}
\end{center}

Spettri raccolti:
\begin{center}
\begin{tabular}{p{3cm}p{4cm}p{2cm}p{3cm}}
\toprule
File name	&$\lambda_\text{range}\text{/}\lambda_\text{centre}$ [nm] 	&AT [s]\footnote{$\text{AT}=\text{acquisition time}$} &$P_n$ [$\si{\milli\bar}$]\\
\midrule
$2702specde7$	&$400-500$	&$0.2$		&$4.32\cdot10^{-4}$\\
$2702specde8$	&$400-500$	&$0.2$		&$4.32\cdot10^{-4}$\\
$2702specde9$	&$400-500$	&$0.2$		&$4.32\cdot10^{-4}$\\
$2702specde10$	&$656$		&$0.05$		&$4.32\cdot10^{-4}$\\
$2702specde11$	&$656$		&$0.05$		&$4.32\cdot10^{-4}$\\
$2702specde12$	&$656$		&$0.05$		&$4.32\cdot10^{-4}$\\
\bottomrule
\end{tabular}
\end{center}

\paragraph*{Potenza microonde $\text{50\%}$} ~\\
Caratteristiche tensione-corrente della sonda di Langmuir:
\begin{center}
\begin{tabular}{p{3cm}p{3cm}}
\toprule
File name	&$P_n$ [$\si{\milli\bar}$]\\
\midrule
$2702lang18$	&$4.42\cdot10^{-4}$\\
$2702lang19$	&$4.46\cdot10^{-4}$\\
\bottomrule
\end{tabular}
\end{center}

Spettri raccolti:
\begin{center}
\begin{tabular}{p{3cm}p{4cm}p{2cm}p{3cm}}
\toprule
File name	&$\lambda_\text{range}\text{/}\lambda_\text{centre}$ [nm] 	&AT [s]\footnote{$\text{AT}=\text{acquisition time}$} &$P_n$ [$\si{\milli\bar}$]\\
\midrule
$2702specde55$	&$400-500$	&$0.2$		&$4.46\cdot10^{-4}$\\
$2702specde56$	&$400-500$	&$0.2$		&$4.46\cdot10^{-4}$\\
$2702specde57$	&$400-500$	&$0.2$		&$4.46\cdot10^{-4}$\\
$2702specde58$	&$656$		&$0.05$		&$4.46\cdot10^{-4}$\\
$2702specde59$	&$656$		&$0.05$		&$4.46\cdot10^{-4}$\\
$2702specde60$	&$656$		&$0.05$		&$4.46\cdot10^{-4}$\\
\bottomrule
\end{tabular}
\end{center}

\paragraph*{Potenza microonde $\text{60\%}$} ~\\
Caratteristiche tensione-corrente della sonda di Langmuir:
\begin{center}
\begin{tabular}{p{3cm}p{3cm}}
\toprule
File name	&$P_n$ [$\si{\milli\bar}$]\\
\midrule
$2702lang04$	&$4.32\cdot10^{-4}$\\
$2702lang05$	&$4.34\cdot10^{-4}$\\
\bottomrule
\end{tabular}
\end{center}

Spettri raccolti:
\begin{center}
\begin{tabular}{p{3cm}p{4cm}p{2cm}p{3cm}}
\toprule
File name	&$\lambda_\text{range}\text{/}\lambda_\text{centre}$ [nm] 	&AT [s]\footnote{$\text{AT}=\text{acquisition time}$} &$P_n$ [$\si{\milli\bar}$]\\
\midrule
$2702specde13$	&$400-500$	&$0.2$		&$4.32\cdot10^{-4}$\\
$2702specde14$	&$400-500$	&$0.2$		&$4.32\cdot10^{-4}$\\
$2702specde15$	&$400-500$	&$0.2$		&$4.32\cdot10^{-4}$\\
$2702specde16$	&$656$		&$0.05$		&$4.32\cdot10^{-4}$\\
$2702specde17$	&$656$		&$0.05$		&$4.32\cdot10^{-4}$\\
$2702specde18$	&$656$		&$0.05$		&$4.32\cdot10^{-4}$\\
\bottomrule
\end{tabular}
\end{center}



\subsection{File utilizzati per pressione di Deuterio a $9.31\cdot10^{-5}mbar$}

\paragraph*{Potenza microonde $\text{20\%}$ (Flusso $9.5sccm$) }~\\
Caratteristiche tensione-corrente della sonda di Langmuir:
\begin{center}
\begin{tabular}{p{3cm}p{3cm}}
\toprule
File name	&$P_{n}$ [$\si{\milli\bar}$]\\
\midrule
$2702lang12$	&$1.20\cdot10^{-4}$\\
$2702lang13$	&$1.18\cdot10^{-4}$\\
\bottomrule
\end{tabular}
\end{center}

Spettri raccolti:
\begin{center}
\begin{tabular}{p{3cm}p{4cm}p{2cm}p{3cm}}
\toprule
File name	&$\lambda_\text{range}\text{/}\lambda_\text{centre}$ [nm] &AT [s]\footnote{$\text{AT}=\text{acquisition time}$} &$P_{n}$ [$\si{\milli\bar}$]\\
\midrule
$2702specde37$	&$400-500$	&$0.2$		&$1.19\cdot10^{-4}$\\
$2702specde38$	&$400-500$	&$0.2$		&$1.19\cdot10^{-4}$\\
$2702specde39$	&$400-500$	&$0.2$		&$1.19\cdot10^{-4}$\\
$2702specde40$	&$656$		&$0.05$		&$1.18\cdot10^{-4}$\\
$2702specde41$	&$656$		&$0.05$		&$1.18\cdot10^{-4}$\\
$2702specde42$	&$656$		&$0.05$		&$1.18\cdot10^{-4}$\\

\bottomrule
\end{tabular}
\end{center}

\paragraph*{Potenza microonde $\text{30\%}$} ~\\
Caratteristiche tensione-corrente della sonda di Langmuir:
\begin{center}
  \begin{tabular}{p{3cm}p{3cm}}
  \toprule
File name	&$P_{n}$ [$\si{\milli\bar}$]\\
  \midrule
$2702lang14$	&$1.16\cdot10^{-4}$\\
$2702lang15$	&$1.18\cdot10^{-4}$\\

  \bottomrule
  \end{tabular}
\end{center}

Spettri raccolti:
\begin{center}
\begin{tabular}{p{3cm}p{4cm}p{2cm}p{3cm}}
\toprule
File name	&$\lambda_\text{range}\text{/}\lambda_\text{centre}$ [nm] &AT [s]\footnote{$\text{AT}=\text{acquisition time}$} &$P_{n}$ [$\si{\milli\bar}$]\\
\midrule
$2702specde43$	&$400-400$	&$0.2$		&$1.17\cdot10^{-4}$\\
$2702specde44$	&$400-400$	&$0.2$		&$1.18\cdot10^{-4}$\\
$2702specde45$	&$400-400$	&$0.2$		&$1.18\cdot10^{-4}$\\
$2702specde46$	&$656$		&$0.05$		&$1.18\cdot10^{-4}$\\
$2702specde47$	&$656$		&$0.05$		&$1.18\cdot10^{-4}$\\
$2702specde48$	&$656$		&$0.05$		&$1.18\cdot10^{-4}$\\

\bottomrule
\end{tabular}
\end{center}

\paragraph*{Potenza microonde $\text{40\%}$} ~\\
Caratteristiche tensione-corrente della sonda di Langmuir:
\begin{center}
\begin{tabular}{p{3cm}p{3cm}}
\toprule
File name	&$P_n$ [$\si{\milli\bar}$]\\
\midrule
$2702lang06$	&$1.04\cdot10^{-4}$\\
$2702lang07$	&$1.08\cdot10^{-4}$\\
\bottomrule
\end{tabular}
\end{center}

Spettri raccolti:
\begin{center}
\begin{tabular}{p{3cm}p{4cm}p{2cm}p{3cm}}
\toprule
File name	&$\lambda_\text{range}\text{/}\lambda_\text{centre}$ [nm] 	&AT [s]\footnote{$\text{AT}=\text{acquisition time}$} &$P_n$ [$\si{\milli\bar}$]\\
\midrule
$2702specde19$	&$400-400$	&$0.2$		&$1.07\cdot10^{-4}$\\
$2702specde20$	&$400-400$	&$0.2$		&$1.07\cdot10^{-4}$\\
$2702specde21$	&$400-400$	&$0.2$		&$1.08\cdot10^{-4}$\\
$2702specde22$	&$656$		&$0.05$		&$1.08\cdot10^{-4}$\\
$2702specde23$	&$656$		&$0.05$		&$1.08\cdot10^{-4}$\\
$2702specde24$	&$656$		&$0.05$		&$1.08\cdot10^{-4}$\\
\bottomrule
\end{tabular}
\end{center}

\paragraph*{Potenza microonde $\text{50\%}$} ~\\
Caratteristiche tensione-corrente della sonda di Langmuir:
\begin{center}
\begin{tabular}{p{3cm}p{3cm}}
\toprule
File name	&$P_n$ [$\si{\milli\bar}$]\\
\midrule
$2702lang08$	&$1.15\cdot10^{-4}$\\
$2702lang09$	&$1.19\cdot10^{-4}$\\
\bottomrule
\end{tabular}
\end{center}

Spettri raccolti:
\begin{center}
\begin{tabular}{p{3cm}p{4cm}p{2cm}p{3cm}}
\toprule
File name	&$\lambda_\text{range}\text{/}\lambda_\text{centre}$ [nm] 	&AT [s]\footnote{$\text{AT}=\text{acquisition time}$} &$P_n$ [$\si{\milli\bar}$]\\
\midrule
$2702specde25$	&$400-400$	&$0.2$		&$1.16\cdot10^{-4}$\\
$2702specde26$	&$400-400$	&$0.2$		&$1.16\cdot10^{-4}$\\
$2702specde27$	&$400-400$	&$0.2$		&$1.16\cdot10^{-4}$\\
$2702specde28$	&$656$		&$0.05$		&$1.16\cdot10^{-4}$\\
$2702specde29$	&$656$		&$0.05$		&$1.18\cdot10^{-4}$\\
$2702specde30$	&$656$		&$0.05$		&$1.18\cdot10^{-4}$\\
\bottomrule
\end{tabular}
\end{center}

\paragraph*{Potenza microonde $\text{60\%}$} ~\\
Caratteristiche tensione-corrente della sonda di Langmuir:
\begin{center}
\begin{tabular}{p{3cm}p{3cm}}
\toprule
File name	&$P_n$ [$\si{\milli\bar}$]\\
\midrule
$2702lang10$	&$1.15\cdot10^{-4}$\\
$2702lang11$	&$1.61\cdot10^{-4}$\\
\bottomrule
\end{tabular}
\end{center}

Spettri raccolti:
\begin{center}
\begin{tabular}{p{3cm}p{4cm}p{2cm}p{3cm}}
\toprule
File name	&$\lambda_\text{range}\text{/}\lambda_\text{centre}$ [nm] 	&AT [s]\footnote{$\text{AT}=\text{acquisition time}$} &$P_n$ [$\si{\milli\bar}$]\\
\midrule
$2702specde31$	&$400-400$	&$0.2$		&$1.49\cdot10^{-4}$\\
$2702specde32$	&$400-400$	&$0.2$		&$1.49\cdot10^{-4}$\\
$2702specde33$	&$400-400$	&$0.2$		&$1.50\cdot10^{-4}$\\
$2702specde34$	&$656$		&$0.05$		&$1.55\cdot10^{-4}$\\
$2702specde35$	&$656$		&$0.05$		&$1.60\cdot10^{-4}$\\
$2702specde36$	&$656$		&$0.05$		&$1.61\cdot10^{-4}$\\
\bottomrule
\end{tabular}
\end{center}

\backmatter

\end{document}