\chapter{Calcolo della densità e della temperatura elettronica per un plasma di Deuterio attraverso due differenti tecniche: sonda di Langmuir e modello spettroscopico}
\chaptermark{Plasma di Deuterio}


L'obiettivo di questo esperimento è il medesimo dell'esperimento precedente ma questa volta lo studio viene effettuato su un plasma di Deuterio.

\section{Strumenti utilizzati e il loro funzionamento: sonda di Langmuir e spettroscopio ottico}
Per descrizione dettagliata degli strumenti vedi capitolo precedente, \textit{Sezione}~\ref{sec:strumenti}.

\section{Condizioni sperimentali e analisi dati}
Le condizioni sperimentali utilizzate sono le seguenti:
\begin{itemize}
 \item corrente delle bobine di GyM fissa a $600\si{\ampere}$;
 \item due differenti pressioni di Deuterio, $4.93\cdot10^{-4}mbar$ (circa $45sccm$ di flusso) e $1.11\cdot10^{-4}mbar$ (circa $10sccm$ di flusso);
 \item diverse potenze per le microonde, $20\%$, $30\%$, $40\%$, $50\%$ e $60\%$.
\end{itemize}


\paragraph{Osservazioni} ~\\
%OSSERVAZIONI A CASO: noi abbiamo fatto rapporto tra i X_alpha/X_delta perchè tutti gli altri davano risultati non compresi tra 1<T<10, e e^9<n_E<e^11 e bo...non ci dà neanche valori qui. 
%Per Lavrov(che ci dà i valori di X idipendenti da n_e) calcoliamo perà H_alpha/H_beta perchè sono i valori che abbiamo tabulati.

A differenza dell'esperimento precedente, si è deciso di mantenere costanti le due pressioni di gas a plasma spento, con valori scelti di $4.93\cdot10^{-4}mbar$ e $1.11\cdot10^{-4}mbar$,
piuttosto che mantenerne fisso il flusso. 
Per mantenere le pressioni tali si sono operate a volte delle leggere variazioni di flusso di Deuterio.\\
Le variazioni di potenza delle microonde sono state effettuate riportando di volta in volta il gas alla condizione di "non-plasma" e aspettando lo stabilizzarsi della pressione ai due valori
indicati predentemente per uno e l'altro caso.\\
Gli accorgimenti di qui sopra sono stati fatti per mantenere le condizioni del sistema il più possibile invariate, così da poter confrontare i dati raccolti a diversi flussi e potenze in modo consistente.\\
L'accensione del plasma, nel caso di plasma di Deuterio, è visibile anche attraverso una discesa seguita da una salita della pressione in camera che raggiunge un \textit{plateau} dopo qualche secondo dall'accensione delle microonde.
A bassa pressione e bassa potenza delle microonde, $1.11\cdot10^{-4}$ con $20\%$ di potenza, si è visto che l'accensione del plasma avviene senza un abbassamento precedente alla salita della pressione.

\subsection{Analisi dati: profilo di densità a diverse pressioni e potenze}

GRAFICI E COMMENTI
\subsection{Analisi dati: profili di temperatura a diverse pressioni e potenze}

GRAFICI E COMMENTI

\subsection{Descrizione del programma utilizzato per l'analisi dati}
In Allegato(VA FATTO RIFERIMENTO) viene riportato il programma scritto per l'analisi dei dati.\\
DESCRIVIAMO A GRANDI LINEE COSA FA


\section{Conclusioni finali}
BLA BLA

\section{File dati}
\subsection{File utilizzati per pressione di Deuterio a $4.93\cdot10^{-4}mbar$}

\paragraph*{Potenza microonde $\text{20\%}$} ~\\
Caratteristiche tensione-corrente della sonda di Langmuir:
\begin{center}
\begin{tabular}{p{3cm}p{3cm}}
\toprule
File name	&$P_{n}$ [$\si{\milli\bar}$]\\
\midrule
$2702lang01$	&$4.95\cdot10^{-4}$\\
\bottomrule
\end{tabular}
\end{center}

Spettri raccolti:
\begin{center}
\begin{tabular}{p{3cm}p{4cm}p{2cm}p{3cm}}
\toprule
File name	&$\lambda_\text{range}\text{/}\lambda_\text{centre}$ [nm] &AT [s]\footnote{$\text{AT}=\text{acquisition time}$} &$P_{n}$ [$\si{\milli\bar}$]\\
\midrule
$2702specde1$	&$400-500$	&$0.2$		&$4.98\cdot10^{-4}$\\
$2702specde2$	&$400-500$	&$0.2$		&$4.98\cdot10^{-4}$\\
$2702specde3$	&$400-500$	&$0.2$		&$4.98\cdot10^{-4}$\\
$2702specde4$	&$656$		&$0.05$		&$4.98\cdot10^{-4}$\\
$2702specde5$	&$656$		&$0.05$		&$4.98\cdot10^{-4}$\\
$2702specde6$	&$656$		&$0.05$		&$4.98\cdot10^{-4}$\\

\bottomrule
\end{tabular}
\end{center}

\paragraph*{Potenza microonde $\text{30\%}$} ~\\
Caratteristiche tensione-corrente della sonda di Langmuir:
\begin{center}
  \begin{tabular}{p{3cm}p{3cm}}
  \toprule
File name	&$P_{n}$ [$\si{\milli\bar}$]\\
  \midrule
$2702lang16$	&$5.32\cdot10^{-4}$\\
$2702lang17$	&$5.32\cdot10^{-4}$\\

  \bottomrule
  \end{tabular}
\end{center}

Spettri raccolti:
\begin{center}
\begin{tabular}{p{3cm}p{4cm}p{2cm}p{3cm}}
\toprule
File name	&$\lambda_\text{range}\text{/}\lambda_\text{centre}$ [nm] &AT [s]\footnote{$\text{AT}=\text{acquisition time}$} &$P_{n}$ [$\si{\milli\bar}$]\\
\midrule
$2702specde49$	&$400-500$	&$0.2$		&$5.33\cdot10^{-4}$\\
$2702specde50$	&$400-500$	&$0.2$		&$5.32\cdot10^{-4}$\\
$2702specde51$	&$400-500$	&$0.2$		&$5.32\cdot10^{-4}$\\
$2702specde52$	&$656$		&$0.05$		&$5.32\cdot10^{-4}$\\
$2702specde53$	&$656$		&$0.05$		&$5.32\cdot10^{-4}$\\
$2702specde54$	&$656$		&$0.05$		&$5.32\cdot10^{-4}$\\

\bottomrule
\end{tabular}
\end{center}

\paragraph*{Potenza microonde $\text{40\%}$} ~\\
Caratteristiche tensione-corrente della sonda di Langmuir:
\begin{center}
\begin{tabular}{p{3cm}p{3cm}}
\toprule
File name	&$P_n$ [$\si{\milli\bar}$]\\
\midrule
$2702lang02$	&$4.85\cdot10^{-4}$\\
$2702lang03$	&$5.13\cdot10^{-4}$\\
\bottomrule
\end{tabular}
\end{center}

Spettri raccolti:
\begin{center}
\begin{tabular}{p{3cm}p{4cm}p{2cm}p{3cm}}
\toprule
File name	&$\lambda_\text{range}\text{/}\lambda_\text{centre}$ [nm] 	&AT [s]\footnote{$\text{AT}=\text{acquisition time}$} &$P_n$ [$\si{\milli\bar}$]\\
\midrule
$2702specde7$	&$400-500$	&$0.2$		&$5.13\cdot10^{-4}$\\
$2702specde8$	&$400-500$	&$0.2$		&$5.13\cdot10^{-4}$\\
$2702specde9$	&$400-500$	&$0.2$		&$5.13\cdot10^{-4}$\\
$2702specde10$	&$656$		&$0.05$		&$5.13\cdot10^{-4}$\\
$2702specde11$	&$656$		&$0.05$		&$5.13\cdot10^{-4}$\\
$2702specde12$	&$656$		&$0.05$		&$5.13\cdot10^{-4}$\\
\bottomrule
\end{tabular}
\end{center}

\paragraph*{Potenza microonde $\text{50\%}$} ~\\
Caratteristiche tensione-corrente della sonda di Langmuir:
\begin{center}
\begin{tabular}{p{3cm}p{3cm}}
\toprule
File name	&$P_n$ [$\si{\milli\bar}$]\\
\midrule
$2702lang18$	&$5.24\cdot10^{-4}$\\
$2702lang19$	&$5.30\cdot10^{-4}$\\
\bottomrule
\end{tabular}
\end{center}

Spettri raccolti:
\begin{center}
\begin{tabular}{p{3cm}p{4cm}p{2cm}p{3cm}}
\toprule
File name	&$\lambda_\text{range}\text{/}\lambda_\text{centre}$ [nm] 	&AT [s]\footnote{$\text{AT}=\text{acquisition time}$} &$P_n$ [$\si{\milli\bar}$]\\
\midrule
$2702specde55$	&$400-500$	&$0.2$		&$5.30\cdot10^{-4}$\\
$2702specde56$	&$400-500$	&$0.2$		&$5.30\cdot10^{-4}$\\
$2702specde57$	&$400-500$	&$0.2$		&$5.30\cdot10^{-4}$\\
$2702specde58$	&$656$		&$0.05$		&$5.30\cdot10^{-4}$\\
$2702specde59$	&$656$		&$0.05$		&$5.30\cdot10^{-4}$\\
$2702specde60$	&$656$		&$0.05$		&$5.30\cdot10^{-4}$\\
\bottomrule
\end{tabular}
\end{center}

\paragraph*{Potenza microonde $\text{60\%}$} ~\\
Caratteristiche tensione-corrente della sonda di Langmuir:
\begin{center}
\begin{tabular}{p{3cm}p{3cm}}
\toprule
File name	&$P_n$ [$\si{\milli\bar}$]\\
\midrule
$2702lang04$	&$5.13\cdot10^{-4}$\\
$2702lang05$	&$5.15\cdot10^{-4}$\\
\bottomrule
\end{tabular}
\end{center}

Spettri raccolti:
\begin{center}
\begin{tabular}{p{3cm}p{4cm}p{2cm}p{3cm}}
\toprule
File name	&$\lambda_\text{range}\text{/}\lambda_\text{centre}$ [nm] 	&AT [s]\footnote{$\text{AT}=\text{acquisition time}$} &$P_n$ [$\si{\milli\bar}$]\\
\midrule
$2702specde13$	&$400-500$	&$0.2$		&$5.13\cdot10^{-4}$\\
$2702specde14$	&$400-500$	&$0.2$		&$5.13\cdot10^{-4}$\\
$2702specde15$	&$400-500$	&$0.2$		&$5.13\cdot10^{-4}$\\
$2702specde16$	&$656$		&$0.05$		&$5.13\cdot10^{-4}$\\
$2702specde17$	&$656$		&$0.05$		&$5.13\cdot10^{-4}$\\
$2702specde18$	&$656$		&$0.05$		&$5.13\cdot10^{-4}$\\
\bottomrule
\end{tabular}
\end{center}



\subsection{File utilizzati per pressione di Deuterio a $1.11\cdot10^{-4}mbar$}

\paragraph*{Potenza microonde $\text{20\%}$ (Flusso $9.5sccm$) }~\\
Caratteristiche tensione-corrente della sonda di Langmuir:
\begin{center}
\begin{tabular}{p{3cm}p{3cm}}
\toprule
File name	&$P_{n}$ [$\si{\milli\bar}$]\\
\midrule
$2702lang12$	&$1.42\cdot10^{-4}$\\
$2702lang13$	&$1.40\cdot10^{-4}$\\
\bottomrule
\end{tabular}
\end{center}

Spettri raccolti:
\begin{center}
\begin{tabular}{p{3cm}p{4cm}p{2cm}p{3cm}}
\toprule
File name	&$\lambda_\text{range}\text{/}\lambda_\text{centre}$ [nm] &AT [s]\footnote{$\text{AT}=\text{acquisition time}$} &$P_{n}$ [$\si{\milli\bar}$]\\
\midrule
$2702specde37$	&$400-500$	&$0.2$		&$1.41\cdot10^{-4}$\\
$2702specde38$	&$400-500$	&$0.2$		&$1.41\cdot10^{-4}$\\
$2702specde39$	&$400-500$	&$0.2$		&$1.41\cdot10^{-4}$\\
$2702specde40$	&$656$		&$0.05$		&$1.40\cdot10^{-4}$\\
$2702specde41$	&$656$		&$0.05$		&$1.40\cdot10^{-4}$\\
$2702specde42$	&$656$		&$0.05$		&$1.40\cdot10^{-4}$\\
\bottomrule
\end{tabular}
\end{center}

\paragraph*{Potenza microonde $\text{30\%}$} ~\\
Caratteristiche tensione-corrente della sonda di Langmuir:
\begin{center}
  \begin{tabular}{p{3cm}p{3cm}}
  \toprule
File name	&$P_{n}$ [$\si{\milli\bar}$]\\
  \midrule
$2702lang14$	&$1.38\cdot10^{-4}$\\
$2702lang15$	&$1.40\cdot10^{-4}$\\
  \bottomrule
  \end{tabular}
\end{center}

Spettri raccolti:
\begin{center}
\begin{tabular}{p{3cm}p{4cm}p{2cm}p{3cm}}
\toprule
File name	&$\lambda_\text{range}\text{/}\lambda_\text{centre}$ [nm] &AT [s]\footnote{$\text{AT}=\text{acquisition time}$} &$P_{n}$ [$\si{\milli\bar}$]\\
\midrule
$2702specde43$	&$400-500$	&$0.2$		&$1.39\cdot10^{-4}$\\
$2702specde44$	&$400-500$	&$0.2$		&$1.40\cdot10^{-4}$\\
$2702specde45$	&$400-500$	&$0.2$		&$1.40\cdot10^{-4}$\\
$2702specde46$	&$656$		&$0.05$		&$1.40\cdot10^{-4}$\\
$2702specde47$	&$656$		&$0.05$		&$1.40\cdot10^{-4}$\\
$2702specde48$	&$656$		&$0.05$		&$1.40\cdot10^{-4}$\\

\bottomrule
\end{tabular}
\end{center}

\paragraph*{Potenza microonde $\text{40\%}$} ~\\
Caratteristiche tensione-corrente della sonda di Langmuir:
\begin{center}
\begin{tabular}{p{3cm}p{3cm}}
\toprule
File name	&$P_n$ [$\si{\milli\bar}$]\\
\midrule
$2702lang06$	&$1.24\cdot10^{-4}$\\
$2702lang07$	&$1.28\cdot10^{-4}$\\
\bottomrule
\end{tabular}
\end{center}

Spettri raccolti:
\begin{center}
\begin{tabular}{p{3cm}p{4cm}p{2cm}p{3cm}}
\toprule
File name	&$\lambda_\text{range}\text{/}\lambda_\text{centre}$ [nm] 	&AT [s]\footnote{$\text{AT}=\text{acquisition time}$} &$P_n$ [$\si{\milli\bar}$]\\
\midrule
$2702specde19$	&$400-500$	&$0.2$		&$1.27\cdot10^{-4}$\\
$2702specde20$	&$400-500$	&$0.2$		&$1.27\cdot10^{-4}$\\
$2702specde21$	&$400-500$	&$0.2$		&$1.28\cdot10^{-4}$\\
$2702specde22$	&$656$		&$0.05$		&$1.28\cdot10^{-4}$\\
$2702specde23$	&$656$		&$0.05$		&$1.28\cdot10^{-4}$\\
$2702specde24$	&$656$		&$0.05$		&$1.28\cdot10^{-4}$\\
\bottomrule
\end{tabular}
\end{center}

\paragraph*{Potenza microonde $\text{50\%}$} ~\\
Caratteristiche tensione-corrente della sonda di Langmuir:
\begin{center}
\begin{tabular}{p{3cm}p{3cm}}
\toprule
File name	&$P_n$ [$\si{\milli\bar}$]\\
\midrule
$2702lang08$	&$1.37\cdot10^{-4}$\\
$2702lang09$	&$1.41\cdot10^{-4}$\\
\bottomrule
\end{tabular}
\end{center}

Spettri raccolti:
\begin{center}
\begin{tabular}{p{3cm}p{4cm}p{2cm}p{3cm}}
\toprule
File name	&$\lambda_\text{range}\text{/}\lambda_\text{centre}$ [nm] 	&AT [s]\footnote{$\text{AT}=\text{acquisition time}$} &$P_n$ [$\si{\milli\bar}$]\\
\midrule
$2702specde25$	&$400-500$	&$0.2$		&$1.38\cdot10^{-4}$\\
$2702specde26$	&$400-500$	&$0.2$		&$1.38\cdot10^{-4}$\\
$2702specde27$	&$400-500$	&$0.2$		&$1.38\cdot10^{-4}$\\
$2702specde28$	&$656$		&$0.05$		&$1.38\cdot10^{-4}$\\
$2702specde29$	&$656$		&$0.05$		&$1.40\cdot10^{-4}$\\
$2702specde30$	&$656$		&$0.05$		&$1.40\cdot10^{-4}$\\
\bottomrule
\end{tabular}
\end{center}

\paragraph*{Potenza microonde $\text{60\%}$} ~\\
Caratteristiche tensione-corrente della sonda di Langmuir:
\begin{center}
\begin{tabular}{p{3cm}p{3cm}}
\toprule
File name	&$P_n$ [$\si{\milli\bar}$]\\
\midrule
$2702lang10$	&$1.72\cdot10^{-4}$\\
$2702lang11$	&$1.90\cdot10^{-4}$\\
\bottomrule
\end{tabular}
\end{center}

Spettri raccolti:
\begin{center}
\begin{tabular}{p{3cm}p{4cm}p{2cm}p{3cm}}
\toprule
File name	&$\lambda_\text{range}\text{/}\lambda_\text{centre}$ [nm] 	&AT [s]\footnote{$\text{AT}=\text{acquisition time}$} &$P_n$ [$\si{\milli\bar}$]\\
\midrule
$2702specde31$	&$400-500$	&$0.2$		&$1.77\cdot10^{-4}$\\
$2702specde32$	&$400-500$	&$0.2$		&$1.77\cdot10^{-4}$\\
$2702specde33$	&$400-500$	&$0.2$		&$1.78\cdot10^{-4}$\\
$2702specde34$	&$656$		&$0.05$		&$1.84\cdot10^{-4}$\\
$2702specde35$	&$656$		&$0.05$		&$1.90\cdot10^{-4}$\\
$2702specde36$	&$656$		&$0.05$		&$1.90\cdot10^{-4}$\\
\bottomrule
\end{tabular}
\end{center}